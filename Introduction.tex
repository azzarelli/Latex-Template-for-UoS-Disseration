%% ----------------------------------------------------------------
%% Introduction.tex
%% ---------------------------------------------------------------- 
\chapter{Introduction} \label{ch:intro}
\section{Introduction}\label{ch:intro.intro}
The goal of this project is to propose a philosophical framework that incorporates artificial intelligence methods with the aim of assisting a (human) \textit{client} in making a decision given an ethical dilemma. To accomplish this, several intermediary decisions regarding the potential outcomes will be made relative to selected philosophical perspectives, for example the \textit{moral perspective} will supply a morally-just decision, whereas the \textit{social perspective} will provide a socially-just decision. Thereafter amalgamating decisions, a final (single) decision will be determined reasonably, thus the assistant provides the client with a suitable choice of outcome. 

The two prevalent topics of research explored within this report consist of, the subject of philosophy investigated in Chapter~\ref{ch:philo}, which outlines the philosophical properties that my decision-making (DM) tool will incorporate, and technological modelling explored in Chapter~\ref{ch:tech}, which underpins the DM techniques I have applied to my decision-making model.

Much of the (philosophical) literature around decision making focuses on several branches of philosophy: (1)\textit{Meta-ethics}, the study of morality, as \cite{miner2003moral} outlines there is a "need for training in moral philosophy"\footnote{In the context of the impact of \textit{Moral Theory} on decision making models}, (2)\textit{Social influence}, described by \cite{baqer2012tech} as the ethics that "change with the trends of society", and (3)\textit{Legal influence}, which is acknowledged by \cite{dennis2013ethical} as important in any autonomous system, though can be disregarded when illegalities can not be avoided or when other ethics take precedent. I have labelled these (Metaethics, Sociology and Legality) as the \textbf{pillars} to my DM model, where (2) holds the most weight, given a study from \cite{nolan2008normative} supports\textit{ descriptive normative beliefs}\footnote{\textit{Descriptive normative beliefs} refer to what an individual think the social norm is} as the most influential behaviour\footnote{Interestingly the study exhibits peoples absent-thought towards the influence of normative action} and (1) would hold the second rank, as I have reasoned with \cite{dennis2013ethical} behind the lowered importance of legal impact.

In addition to layering my model with philosophical reasoning, I need to apply a technical framework that supports it. There are several popular structures of DM models, such as \textit{priority queues}\footnote{A \textit{priority queue} is a selection sorting algorithm that assigns priorities to each outcome and decides on the highest-priority item}, used in \cite{islam2018algorithm} as a \textit{heap} (tree-like structure), or \textit{cost structures}\footnote{A \textit{cost structure} is a model that incorporates a cost-function with the purpose of finding the least-cost outcome}, explored in \cite{lee1999mathematical} as a \textit{centralised system}\footnote{A \textit{centralised system} incorporates a central component that relays with external components individually. The alternative is a decentralised system,  where all components relay with each other (similar to a fully-connected graph), though the central component still makes the primary decisions}. There also exists purely learning structures (relating to artificial intelligence (AI)), such as in \cite{noothigattu2017voting}, who uses a voting-based system to train decisions, and \cite{KARTAL2016599} who finds algorithms such as support vector machine (SVM) algorithms useful for multi-criteria decision making tasks in cars.

Aside from structuring my framework, I need to hurdle two obstacles in order to develop an exemplar program. The first focuses on the translation of philosophical topics and contextual information\footnote{\textit{Contextual information} refers to information that can not be calculated using a deterministic approach, which allows a situation to be given context. Therefore acts as the source of information that need to be quantified to make a contextual-determination} into language that can be understood by a program. The second focuses on the reduction of bias when evaluating testing data\footnote{\textit{Testing data} refers to the data used to test a machine learning model that has been trained using \textit{training data}, by extension test data $\neq$ training data}. Even though these topics are not explored in this report, they are still worth mentioning as they will be my primary focus in the forthcoming months, as outlined in my Gantt Char, Appendix~\ref{fig:gantt}, and in Chapter~\ref{ch:aow.plan}.

Otherwise, the structure of this report is as follows. Chapter~\ref{ch:research} will focus on the body of research. More precisely Chapter~\ref{ch:philo} will focus on the philosophical reasoning implemented in my DM model, whilst in Chapter~\ref{ch:tech}, I will apply DM and learning techniques in order to develop an exemplar program. In Chapter~\ref{ch:prop} I propose the finalised philosophical framework and in and Chapter~\ref{ch:aow} I aim to emphasise the work done and strictly layout the plan going forward.

\section{Further Insight}\label{ch:intro.plan}
The goal is to assist a client in choosing the solution from a list of reasonable solution, that will yield the \textit{best}\footnote{The use of \textit{best} suggests a layer of subjectivity to our model - this is a common flaw that many decision-making robots succumb to. I must recognise that it is an inevitable flaw within my model as my choice to (or not to) include certain philosophies means my robot will be constrained to \textit{my} subjectivity, though I aim to generalise my solution so that the implication of my subjectivity in this manner is minimised} result. More specifically the aim is to develop a philosophical framework with applied DM and ML techniques in order to assist a cleints decision making process in an ethical dilemma, and further illustrate it's workings by use of program. The plan outlined by the Gantt Chart follows the development of the philosophical model and a corresponding program. My research methodology can be found in Chapter~\ref{ch:aow.plan}.

To briefly comment on the impact of COVID-19: there is little to fret about the construction of my model; whether philosophical or technological, is not constrained by any physical instruments or facilities, nor is/will there be any need for person-to-person contact.